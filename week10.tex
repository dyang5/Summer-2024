Monday, July 22nd
\begin{itemize}
    \item Came up with conjecture for $F_p^k$ terms (extending Lenart Maeno) in Grassmanians
    \item Confirmed conjecture for $Gr(1, 5), Gr(2, 5), Gr(3, 5), Gr(4, 5)$ and the corresponding Chevalley relations (very excited as this could be new!)
\end{itemize}

Tuesday, July 23rd
\begin{itemize}
    \item Met with Linda to discuss progress on guesses/Grassmanians and next steps (testing in larger ambient spaces/automating the process)
    \item Discovered first working guesses for $F_p^k$ in $Fl(1, 2, 4)$ and $Fl(1, 3, 4)$ -- all Chevalley satisfied!
    \item Next step: test Grassmanian conjecture against $Gr(3, 6), Gr(4, 7), Gr(4, 8)$ (create way to automate this process), and above guesses for $Fl(2, 3, 4)$ and other partial flags (again, find a way to automate this process).
\end{itemize}

Wednesday, July 24th
\begin{itemize}
    \item Working on automizing Chevalley relation tests
    \item Coded up $F_p^k$ conjecture for any partial flag, and corresponding Grothendieck functions
    \item Checked that relations are satisfied for $Fl(1, 2, 3, 5)$
\end{itemize}

Thursday. July 25th
\begin{itemize}
    \item Further work on optimizing Chevalley checking
    \item Possible next steps:
    \item Recursions from Lenart-Maeno -- recursions can give you Pieri rules
    \item We can also show the presentations are consistent w Wu/Sharpe/Xu. Hopefully their recursions match up w our recursions. Recursions may or may not be enough to prove Chevalley
    \item So...have recursions on the f's. Use that to choose they fit the relations.
    \item Zoe: work on relations. Also come up with a nice looking definition of the F's. F(p,n,k) in terms of F(p,n-1,k). 
\end{itemize}

