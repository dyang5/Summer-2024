
%--------------------------------------------------------------------------------
\documentclass[12pt]{amsart}
\usepackage{etex}
%\usepackage{amssymb}
% \usepackage[foot]{amsaddr}
\usepackage{graphicx}
\usepackage{epstopdf}
\usepackage{subcaption}

\usepackage{cite}
%\usepackage{tikz, tikz-cd, tkz-graph}
%\usetikzlibrary{arrows, patterns}


\usepackage[margin=1in]{geometry}
\usepackage{times}

\raggedbottom
\usepackage{xcolor}

\usepackage{hyperref}
\hypersetup{
  colorlinks,
  linkcolor={red!50!black},
  citecolor={blue!50!black},
  urlcolor={blue!80!black}
}

\usepackage[english]{babel}
\usepackage{amsmath,amssymb,amsthm}
\usepackage{xypic}
\usepackage{longtable}
\usepackage{array}

 \usepackage[enableskew]{youngtab}

\usepackage{epsfig}
\usepackage{hyperref}
\usepackage{enumitem}
\usepackage{booktabs}
 \usepackage{longtable} %used by appendix
 \usepackage{pdflscape} % used by appendix
 \usepackage{colortbl} % used by appendix
 \usepackage{arydshln} % used by appendix
 \usepackage{calc} % used by appendix
%\usepackage[vcentermath]{genyoungtabtikz}
\Yboxdim{6pt}

\graphicspath{{./images/}}

%--------------------------------------------------------------------------------
% Theorems etc.
%--------------------------------------------------------------------------------





\newtheorem{thm}{\bf Theorem}[section]
\newtheorem{eg}[thm]{\bf Example}
\newtheorem{prop}[thm]{\bf Proposition}
\newtheorem{cor}[thm]{\bf Corollary}
%\newtheorem{rem}[thm]{\bf Remark}
\newtheorem{mydef}[thm]{\bf Definition}
\newtheorem{lem}[thm]{\bf Lemma}
\newtheorem{conjecture}[thm]{\bf Conjecture}


\newtheorem{theorem}{\bf Theorem}
\renewcommand{\thetheorem}{\Alph{theorem}}  % Thm A, B, C

\newtheorem{theoremrepeat}{\bf Theorem}
\renewcommand{\thetheoremrepeat}{\Alph{theoremrepeat}}  % Thm A, B, C

\theoremstyle{remark}
\newtheorem{rem}[thm]{\bf Remark}

%\renewcommand{\thethm}{\Alph{thm}}  % Thm A, B, C



\newcommand{\CC}{\mathbb{C}}
\newcommand{\RR}{\mathbb{R}}
\newcommand{\ZZ}{\mathbb{Z}}
\newcommand{\QQ}{\mathbb{Q}}
\newcommand{\NN}{\mathbb{N}}
\newcommand{\PP}{\mathbb{P}}
\newcommand{\LL}{\mathbb{L}}
\newcommand{\Cstar}{\C^ }
\newcommand{\br}{\mathbf{r}}
\newcommand{\be}{\mathbf{e}}
\newcommand{\bp}{\mathbf{p}}
\newcommand{\bq}{\mathbf{q}}
\newcommand{\bt}{\mathbf{t}}
\newcommand{\one}{\mathbf{1}}
\newcommand{\numberofnewFanos}{$141$}
\newcommand{\mf}{\mathfrak}
\newcommand{\U}{\"u}
\newcommand{\flag}{\Fl(n;\mathbf{r})}
\newcommand{\QH}{\mathrm{QH}}
\newcommand{\HH}{\mathrm{H}}
\newcommand{\QK}{\mathrm{QK}}
\newcommand{\K}{\mathrm{K}}

\newcommand{\cS}{\mathcal{S}}


\newcommand{\nn}{\mathbf{n}}   % index for partial flags

\newcommand{\Sn}{S(n;\mathbf{r})}
\newcommand{\Pn}{P(n;\mathbf{r})}
\newcommand{\rr}{\mathbf{r}}
\newcommand{\Mbar}{\bar{M}}     % Kontsevich space
\newcommand{\ev}{\mathrm{ev}}   % evaluation map


\newcommand{\qq}{\mathbf{q}}   % quantum variable (vector)
\newcommand{\dd}{\mathbf{d}}   % (multi-)degree
\newcommand{\Sch}{\mathfrak{S}} % Schubert polynomial

\DeclareMathOperator{\sgn}{sgn}
\DeclareMathOperator{\Rep}{\mathrm{Rep}}
\DeclareMathOperator{\Spec}{\mathrm{Spec}}
\DeclareMathOperator{\Hom}{\mathrm{Hom}}
\DeclareMathOperator{\GL}{\mathrm{GL}}
\DeclareMathOperator{\Fl}{\mathrm{Fl}}
\DeclareMathOperator{\OG}{\mathrm{OGr}}
\DeclareMathOperator{\Gr}{\mathrm{Gr}}
\DeclareMathOperator{\Sym}{\mathrm{Sym}}
\DeclareMathOperator{\rank}{\mathrm{rank}}
\DeclareMathOperator{\rk}{\mathrm{rk}}
\DeclareMathOperator{\Cox}{\mathrm{Cox}}
\DeclareMathOperator{\NE}{\mathrm{NE}}
\DeclareMathOperator{\NC}{\mathrm{NC}}
\DeclareMathOperator{\Pic}{\mathrm{Pic}}
\DeclareMathOperator{\Amp}{\mathrm{Amp}}
\DeclareMathOperator{\Irr}{\mathrm{Irr}}
\DeclareMathOperator{\Mat}{\mathrm{Mat}}
\DeclareMathOperator{\HR}{\mathrm{RH}}

\DeclareMathOperator{\image}{\mathrm{Im}}
\newcommand{\ab}{\text{\rm ab}}
\renewcommand{\emptyset}{\varnothing}
\newcommand{\intrinsic}[1]{\texttt{\upshape#1}}
\newtheorem{conj}{Conjecture}[section]

\newtheorem{result}[conj]{Result}
\newcolumntype{C}[1]{>{\centering\let\newline\\\arraybackslash\hspace{0pt}}m{#1}}
\newcommand{\F}[2]{\Gamma_{#1}^{#2}}
\newcommand{\TTau}[2]{\gamma_{#1}^{#2}}



\title[]{Quantum Grothendieck polynomials for partial flag varieties}

\author[L.~Chen]{L.~Chen}
\address{Linda Chen \newline \indent Department of Mathematics and Statistics, Swarthmore College, Swarthmore, PA 19081}
\email{lchen@swarthmore.edu}

\author[Z.~Markman]{Z.~Markman}
\address{Zoe Markman \newline \indent Department of Mathematics and Statistics, Swarthmore College, Swarthmore, PA 19081}
\email{zmarkma1@swarthmore.edu.edu}

\author[D.~Yang]{D.~Yang}
\address{David Yang\newline \indent xxx}
\email{xxx}

\thanks{LC, ZM, and DY were partially supported by NSF Grant DMS-2101861. }

\begin{document}
\begin{abstract}
Given a flag variety $\Fl(n_1,\ldots,n_l;n)$ xxx
\end{abstract}

\maketitle

\section{Introduction}\label{sec:intro}
Fix an $n$-dimensional vector space $V$ and a tuple of integers $\nn = (0<n_1<\cdots<n_m<n_{m+1}=n)$, and let $\Fl(\nn) = \Fl(n_1,\ldots,n_m;V)$ denote the partial flag variety parametrizing flags $W_1\subset \cdots \subset W_m \subset V = W_{m+1}$ with $\dim W_p = n_p$. 

\section{Quantum $\K$-theory}
 
 \section{QK Whitney relations}
 


(Conjectural) QK Whitney relations for the equivariant quantum K-theory rings of partial flag varieties were given in  \cite{gs1,gs2}; they were  proved in the case of incidence varieties $\Fl(1,n-1;n)$ and  proved modulo a certain divisor assumption in the case of the complete flag variety.

The flag variety comes with a universal sequence of  bundles $\cS_1\hookrightarrow \cdots \hookrightarrow \cS_m\hookrightarrow V_{\Fl(\nn)}$, where $cS_i$ is the vector bundle of rank $n_i$ whose fiber over a point $W_\bullet \in \Fl(\nn)$ is the vector space $W_i$.  The Whitney relations are written in terms of the Hirzebruch class of bundles associated with this sequence.

For an equivariant vector bundle $E$, the Hirzebruch class of $E$ is
$$\lambda_y(E):= 1+y[E]+\cdots + y^{\rk E} [\wedge^{\rk E}E] \in \K_T(X)[y].$$

We first state the conjectural QK Whitney relations of \cite[Conjecture 1.1]{gs2}.

\begin{conjecture}
\label{conj:whitney}
For $1\leq k=1\leq m$, the following relations hold in $\QK_T(\Fl(\nn))$:
\[
\lambda_y(\cS_k)*\lambda_y(\cS_{k+1}/\cS_k) = \lambda_y(\cS_{k+1}) - y^{n_{k+1}-n_k} \frac{q_k}{1-q_k}\det(\cS_{k+1}/\cS_k)*(\lambda_y(\cS_k)-\lambda_y(\cS_{k-1}).
\]
Here, $\lambda_y(\cS_{m+1})=\lambda_y(\CC^n) = \prod_{i=1}^n (1+yT_i)$, where $T_i\in\K_T(\mathrm{pt})$ are given by the decomposition of $\CC^n$ into one dimensional $T$-modules .
\end{conjecture}

In this section, we give relations between $\wedge^p \cS_k$ arising from the conjectural QK Whitney relations. We define $\F{p}{n_k}: = \wedge^p S_k$, and $\TTau{j}{k}:= (1-q_k)^{=1}\wedge^i (\cS_k / \cS_{k-1})$. 

We can then rewrite the QK Whitney conjectural relations as:

\begin{equation*}
\begin{split}
    (1 + y\F{1}{n_j}+ \ldots + y^{n_j}\F{n_j}{n_j})
    \left(1 + y\TTau{1}{j+1}(1-q_{j+1}) + \ldots + y^{n_{j+1}-n_j-1}\TTau{n_{j+1} - n_{j-1}}{j+1}(1-q_{j+1}) \right) \\ 
    = \sum_{p=0}^{n_{j+1}} y^p \F{p}{n_{j+1}}
    -y^{n_{j+1} - n_j} \frac{q_j}{1-q_j}(1-q_{j+1})^{-1}
\TTau{n_{j+1}-n_j}{j+1}
\left( \sum_{p=0}^{n_j} y^p(\F{p}{n_j} 
- \F{p}{n_{j-1}}) \right)
\end{split}
\end{equation*}

We expand this to:
\begin{equation*}
\begin{split}
    \sum_{p=0}^{n_{j+1}} y^p \F{p}{n_{j+1}} &= (1 + y\F{1}{n_j} + \ldots + y^{n_j}\F{n_j}{n_j}) 
    \left(1 + y\TTau{1}{j+1}(1-q_{j+1}) + \ldots + y^{n_{j+1}-n_j-1} \TTau{n_{j+1} - n_{j-1}}{j+1}(1-q_{j+1}) \right) \\
    &+(1+y\F{1}{n_j} + \ldots + y^{n_j}\F{n_j}{n_j})\left(y^{n_{j+1}-n_j}\TTau{n_{j+1}-n_j}{j+1} (1-q_{j+1}) \right) \\
    &+ y^{n_{j+1} - n_j} \frac{(q_j)(1-q_{j+1})}{1-q_j}
\tau_{n_{j+1} - n_j}^{j+1} 
\left( \sum_{p=0}^{n_j} y^p(\F{p}{n_j} 
- \F{p}{n_{j-1}}) \right).
\end{split}
\end{equation*}

Then, 
\begin{equation*}
\begin{split}
    \sum_{p=0}^{n_{j+1}} y^p \F{p}{n_{j+1}} 
    &= (1 + y\F{1}{n_j} + \ldots + y^{n_j}\F{n_j}{n_j}) 
    \left(1 + y\TTau{1}{j+1}(1-q_{j+1}) + \ldots + y^{n_{j+1}-n_j-1} \tau_{n_{j+1} - n_{j-1}}^{j+1}(1-q_{j+1}) \right) \\
    &+ y^{n_{j+1}-n_j}\TTau{n_{j+1}-n_j}{j+1}(1-q_{j+1}) \left((1+y\F{1}{n_j} + \ldots + y^{n_j}\F{n_j}{n_j}) + \frac{(q_j)(1-q_{j+1})}{1-q_j}\sum_{p = 0}^{n_j} y^p(\F{p}{n_j} - \F{p}{n_{j-1}}) \right)
\end{split}
\end{equation*}

Because $\F{0}{n_j} = \F{0}{n_{j-1}}$, we have $\sum_{p = 0}^{n_j} y^p(\F{p}{n_j} - \F{p}{n_{j-1}}) = \sum_{p > 0}^{n_j} y^p(\F{p}{n_j} - \F{p}{n_{j-1}})$. This yields:

\begin{equation*}
\begin{split}
    \sum_{p=0}^{n_{j+1}} y^p \F{p}{n_{j+1}} 
    &= (1 + y\F{1}{n_j} + \ldots + y^{n_j}\F{n_j}{n_j}) 
    \left(1 + y\TTau{1}{j+1}(1-q_{j+1}) + \ldots + y^{n_{j+1}-n_j-1} \TTau{n_{j+1} - n_{j-1}}{j+1}(1-q_{j+1}) \right) \\
    &+ y^{n_{j+1}-n_j}\TTau{n_{j+1}-n_j}{j+1}(1-q_{j+1}) \left(1 + \frac{1}{1-q_j}(y\F{1}{n_j} + \ldots + y^{n_j}\F{n_j}{n_j}) 
    - \frac{q_j}{1-q_j}\sum_{p>0}^{n_j} y^p \F{p}{n_{j-1}}\right) \\
    &= (\sum_{p>0}^{n_j}y^p\F{p}{n_j})\left(1+ \sum_{l=1}^{n_{j+1}-n_j-1}\TTau{l}{j+1}(1-q_j)\right) \\
    &+ y^{n_{j+1}-n_j}\TTau{n_{j+1}-n_j}{j+1}(1-q_{j+1}) \\
    &+ y^{n_{j+1}-n_j}\TTau{n_{j+1}-n_j}{j+1}(1-q_j)\left(\frac{1}{1-q_j}(yF_1^{n_j} + \ldots + y^{n_j}F_{n_j}^{n_j}) - \frac{q_j}{1-q_j} \sum_{p>0}^{n_k} y^p\F{p}{n_{j-1}}\right) \\
    &= (\sum_{p>0}^{n_j}y^p\F{p}{n_j})\left(1+ \sum_{l=1}^{n_{j+1}-n_j-1}\TTau{l}{j+1}(1-q_j)\right) \\
    &+ y^{n_{j+1}-n_j}\TTau{n_{j+1}-n_j}{j+1}(1-q_{j+1}) \\
    &+ y^{n_{j+1}-n_j}\TTau{n_{j+1}-n_j}{j+1}\frac{1-q_{j+1}}{1-q_j}(y(\F{1}{n_j} - q_j\F{1}{n_{j-1}}) + \ldots + y^{n_{j-1}}(\F{n_{j-1}}{n_j} - q_j \F{n_{j-1}}{n_{j-1}}) \\ &+y^{n_{j-1}+1}\F{n_{j-1}+1}{n_j} + 
    \ldots + y^{n_j}\F{n_j}{n_j}) \\
\end{split}
\end{equation*}

By isolating the coefficient of the $y^p$ term, we get the following recursive formula for $F_p^{n_{j+1}}$:
(come back to write up 4 cases)

We define $\F{0}{k} = 1$, and $\F{p}{k}  = 0$ when $p < 0$. We then have the following recursion:

\begin{equation}
\F{p}{n_{j+1}} = \F{p}{n_j} + \sum_{i=1}^{n_{j+1}-n_j-1} \F{p-i}{n_j}\TTau{i}{j+1}(1-q_{j+1}) + \TTau{n_{j+1} - n_j}{j+1}\frac{1-q_{j+1}}{1-q_j} \left(\F{p-n_{j+1}-n_j}{n_j} - q_j  \F{p-n_{j+1}-n_j}{n_{j-1}} \right).
\end{equation}

We provide the following alternate formulation of the recursion (come back to and type up):


\section{Quantum Grothendieck polynomials}
In this section, we give a brief review of the classical single quantum Grothendieck polynomials before introducing the universal Grothendieck polynomials which can specialize to new quantum Grothendieck polynomials defined for any partial flag variety. 

Begin with the longest permutation in $S_n$, $w_0 = n \, n-1 \, \dots 1$. The Grothendieck polynomial of $w_0$ is the same as the Schubert polynomial for $w_0$:
\[
    \mathfrak{G}_{w_0}(x) = x_1^{n-1} x_2^{n-2} \dots x_{n-1}.
\]
Recall the classic divided difference and isobaric divided difference operators, originally introduced in [add citation -- 10, 20 in LeM]. The divided difference operator $\delta_i$ is defined as
\[ 
    \delta_i = \frac{f - s_i(f)}{x_i - x_{i+1}}
\]
where $s_i$ is the transposition of the indices $i$ and $i+1$. The isobaric divided difference operator $\pi_i$ is defined as
\[
    \pi_i = \delta_i(1-x_{i+1}).
\]
The Grothendieck polynomials for $w \in S_n$ are defined as
\begin{align*}
    \mathfrak{G}_w(x) &= \pi_{w^{-1}w_0}(\mathfrak{G}_{w_0}(x)).
\end{align*}

\begin{mydef}
We define the (equivariant) universal Grothendieck polynomial for $w_0$ in $S_n$ as
\[
    \mathfrak{G}_{w_0}(c, y) = \prod\limits_{i=1}^{n-1}\sum\limits_{j=0}^i (-1)^jc_{j, i}(1-y_{n-i})^j,
\]
extending definition 8.2 of \cite{lm} and equation (7) of \cite{fulton} for Grothendiecks.
\end{mydef}

The universal Grothendieck polynomial for any permutation $w$ in $S_n$ is consequently
\[
    \mathfrak{G}_w(c, y) = \pi_{w^{-1}w_0}^{(y)} \mathfrak{G}_{w_0}(c, y)
\]
where the isobaric divided difference operator $\pi_{w^{-1}w_0}^{(y)}$ acts on the $y$-variables only. 

It also follows that the single universal Grothendieck polynomials are obtained by setting the $y$ variables to $0$, or $\mathfrak{G}_w(c) = \mathfrak{G}_w(c, y)$.

We will introduce two new definitions for the specialization of these universal Grothendieck polynomials to equivariant quantum Grothendieck polynomials.

For $Fl(n_1, \dots, n_l; n)$, define 
\[
    \tau_i^k = e_i(1-x_{n_{k-1} + 1}, \dots, 1-x_{n_k})
\]
for $0 \leq i \leq n_k - n_{k-1}$ and $1 \leq k \leq l + 1$, with $\tau_0^{\ast} = 0$.

\begin{mydef}\label{$F_p^{n_k}$}
For $Fl(n_1, \dots, n_l; n)$, define
\[
F_p^{n_k} = \sum\limits_{a_1 + \dots + a_k = p} \tau_{a_1}^1 \cdot \dots \cdot \tau_{a_k}^k \prod\limits_{a_i > \, 0, \, a_{i+1} \neq n_{i+1} - n_i} (1-q_i)
\]
for $1 \leq k \leq l+1$. Remark: this definition extends equation (3.1) of \cite{lm} for any partial flag.
\end{mydef}


\textcolor{red}{The $F_p^{n_k}$ defined above satisfy the QK Whitney relations.
and we use them to define the (equivariant) quantum Grothendieck polynomials:} to specialize from the universal Grothendieck polynomials to the equivariant Grothendieck polynomials, we set $c_{j, i}$ to the $F_j^{n_k}$ defined in \ref{$F_p^{n_k}$} if $n_k \leq j < n_{k+1}$.


\begin{thebibliography}{12}

\bibitem{a} D.~Anderson, {\em Computing torus-equivariant K-theory of singular varieties}, in {\em Algebraic groups: structure and actions}, 1--15, Proc.~Sympos.~Pure Math., 94, Amer. Math. Soc., Providence, RI, 2017.

\bibitem{act} D.~Anderson, L.~Chen, and H.-H.~Tseng, {\em On the quantum K-ring of the flag manifold}, arXiv:1711.08414.


\bibitem{brion} M.~Brion, {\em Equivariant Chow groups for torus actions}, Transform. Groups 2 (1997), no. 3, 225--267.

\bibitem{bcmp3} A.~Buch, P.-E.~Chaput, L.~Mihalcea, and N.~Perrin, {\em A Chevalley formula for the equivariant quantum K-theory of cominuscule varieties}, to appear in  Algebr. Geom., arXiv:1604.07500v2.

\bibitem{bm} A.~Buch and L.~Mihalcea, {\em Quantum K-theory of Grassmannians}, Duke Math.~J. 156 (2011), no. 3, 501--538.

\bibitem{chen} Linda~Chen, ``Quantum cohomology of flag manifolds,'' {\em Adv. Math.} {\bf 174} (2003), no. 1, 1--34.


\bibitem{cf-flags} Ionu\c{t}~Ciocan-Fontanine, ``The quantum cohomology ring of flag varieties,'' {\em Trans. Amer. Math. Soc.} {\bf 351} (1999), no. 7, 2695--2729.

\bibitem{cf-partial} Ionu\c{t}~Ciocan-Fontanine, ``On quantum cohomology rings of partial flag varieties,'' {\em Duke Math. J.} {\bf 98} (1999), no. 3, 485--524.

\bibitem[Fu]{fulton} William~Fulton, ``Universal Schubert polynomials,'' {\em Duke Math. J.} {\bf 96} (1999), no. 3, 575--594.

\bibitem{g} A. Givental, {\em On the WDVV equation in quantum K-theory}, Dedicated to William Fulton on the occasion of his 60th birthday, Michigan Math. J. 48 (2000), 295--304.

\bibitem{gk} A. Givental and B. Kim. {\em Quantum cohomology of flag manifolds and Toda lattices}, Comm. Math. Phys. 168 (1995), 609--641.

\bibitem{gl} A. Givental and Y.-P. Lee, {\em Quantum K-theory on flag manifolds, finite-difference Toda lattices and quantum groups}, Invent. Math. 151 (2003), no. 1, 193--219.

\bibitem{gs1} W. Gu, L. Mihalcea, E. Sharpe, W. Xu, H. Zhang, and H. Zou, {\em Quantum K theory rings of partial flag manifolds}, arXiv:2306.1109.

\bibitem{gs2} W. Gu, L. Mihalcea, E. Sharpe, W. Xu, H. Zhang, and H. Zou, {\em Quantum K Whitney relations for partial flag varieties}, arXiv:2306.1109.

\bibitem{kpsz} P. Koroteev, P. P. Pushkar, A. Smirnov, and A. M. Zeitlin, {\em Quantum K-theory of quiver varieties and many-body systems}, arXiv:1705.10419. 

\bibitem{l} Y.-P. Lee, {\em Quantum K-theory. I. Foundations}, Duke Math. J. 121 (2004), no. 3, 389--424. 

\bibitem{lm} C.~Lenart and T.~Maeno, {\em Quantum Grothendieck polynomials}, arXiv:0608232.



\end{thebibliography}



\end{document}
