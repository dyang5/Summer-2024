Monday, June 24th
\begin{itemize}
    \item Met with Linda, discussed progress in coding quantum schuberts in partial flags (debugging universal schubert code, leaving entries in terms of $c$ entries in Fulton), and next steps for quantum Grothendieck exploration
    \item Lots of coding and debugging -- fix $c$ entries next and compare expressions to desired ones from Ciocan-Fontanine
\end{itemize}

Tuesday, June 25th
\begin{itemize}
    \item Made more progress on quantum schuberts in partial flags
    \begin{itemize}
        \item Set up function to calculate universal schubert polynomial for any permutation
        \item Set up substitution functions for $c$ and $y$ values (setting $c[i, j]$ to correct quantized standard elementary monomial and $y_i$ terms to $0$)
        \item Created function to calculate all permutations with descents only at subset of given indices
        \item Working on setting up basis of quantum schubert polynomials for given partial flag variety
        \item Next step: decompose product in terms of a given basis
    \end{itemize}
\end{itemize}

Wednesday, June 26th
\begin{itemize}
    \item Fixed some functionality in creating quantum schubert basis and substituting $c[i, j]$ values
    \item Begin work on decomposing product in terms of a quantum basis
\end{itemize}

Thursday, June 27th
\begin{itemize}
    \item Not much progress on product decomposition. Trying to switch to Python decomposition function rather than one in QKCalc.
    \item Begin looking and familiarizing ourselves with Weihong's code for partial flags
\end{itemize}

Friday, June 28th
\begin{itemize}
    \item Met with Linda to discuss next steps
    \item Work on coding to set certain $g_i[j]$ terms to $0$ when decomposing products
    \item Try to create table for qkmonk in partial flags
\end{itemize}

Point of Discussion for Friday, June 28th Meeting \& Notes
\begin{itemize}
    \item Weihong's code: Hecke Actions and multiplication, qkmonk inner function, difference between guess vs inner functions for partial flag (GP\_A)
    \item 3 steps on Jamboard: Speed up code -- don't want $g, y$ polynomials, want q polynomials
    \item For Grothendiecks in partial flags, we can guess longest word for grothendieck, and use isobaric divided difference. Then check that the Monk rules are satisfied (if they are, we have likely found a match!)
\end{itemize}