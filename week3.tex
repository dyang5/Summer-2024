\subsection{Weekly Summary}

Monday, June 3rd
\begin{itemize}
    \item Worked through Sheet 3 (Schubert polynomials introduction, Schubert polynomials in $Fl(3)$, relating Schubert polynomials $\mathfrak{S}$ in the basis of Schubert polynomials with partitions and Schur polynomials).
    \item Read potential projects background
\end{itemize}

Tuesday, June 4th
\begin{itemize}
    \item Met with Linda, talked about Sheet 3 (addressing sheet 3 problems and potential projects)
    \item Started working through potential projects warmup problems, including quantum Schubert polynomials and reading through Buch (Quantum Homology)
\end{itemize}

Wednesday, June 5th
\begin{itemize}
    \item Working on warmup problem exercise. Tried Quantum schubert multiplication with Quantum Pieri (not much success), found answer using Quantum Monk (need to check by hand)
    \item Reading Buch and Lenart-Maeno articles
\end{itemize}

Thursday, June 6th
\begin{itemize}
    \item Continue working on warmup problem exercise. Finalized Quantum Monk calculations for Sheet 3 Quantum Schubert cases
    \item Learned more about Grothendieck and quantum Grothendieck polynomials
    \item Begin working on project 2 problems (Chen, Tymoczko and LLMS as resources)
    \item Zoe and David met in the afternoon to discuss progress
\end{itemize}

\subsection{Points of Discussion for June 4th, Tuesday Meeting}

\begin{itemize}
    \item How can we determine a basis of Schubert polynomials for a general flag manifold?
    \item How can we reconcile the Schubert polynomials with the Schurs (which permutations correspond to which partitions)?
    \item For project 2: Which properties are we looking to preserve under the essential set analog?
\end{itemize}

\subsection{Points of Discussion for June 7th, Friday Meeting}

\begin{itemize}
    \item Regarding quantum Grothendieck polynomials, what is a $k$-Bruhat graph? Is this a nicer way to understand condition 3 from Lenart and Maeno? (which conditions correspond to partial flag, which ones correspond to quantum, which ones correspond to Monk vs. Pieri)
    \item How can we understanding Grothendieck and quantum Grothendieck polynomials? Do they have nice definitions like Schuberts ($\beta$ term)
    \item Discussion of Quantum Monk vs. Quantum Pieri and associated conditions for full flag (Buch page 6)

\end{itemize}


\subsubsection{Reconciling (some) Schubert polynomials with Schurs}
When we refer to the complete flag manifold $Fl(1, 2, 3, 4)$, observe that the Schubert polynomials correspond to permutations of $S_4$, and so $S_4$ is in some sense the basis for the flag manifold:
\[
    Fl(1, 2, 3, 4) \leftrightarrow S_4
\]
In general, for a flag manifold $Fl(n_1, n_2, \dots, n)$, we find that the ``basis'' corresponds to a subset of permutations of $S_n$:
\[
    Fl(n_1, n_2, \dots, n) \leftrightarrow \{ w \in S_n \mid w \text{ has descents } \subseteq \{ n_1, n_2, \dots, n\} \}
\]

Take the specific case of $Gr(2, 4) = Fl(2, 4)$. The basis corresponds to permutations 
$w \in S_4 \mid w \text{ has descent possibly at $2$}$; the permutations in said basis are simply $[1\, 2 \mid 3 \,4]$, $[1\,3 \mid 2\,4]$, $[1\, 3 \mid 2\,4]$, $[1\,4 \mid 2\,3]$, $[2\,3 \mid 1\,4]$, $[2\,4 \mid 1\,3]$, and $[3\,4 \mid 1\,2]$. \\

There is an association between the Schubert polynomials corresponding to each of these permutations and an associated Schur polynomial, characterized by the following rule: let $w = w_1 w_2 \dots w_n$. 
\begin{quote}
\textit{If $w_i < w_{i+1}$ for all $i \neq r$ (i.e. there is just one descent at index $r$), $\mathfrak{S}_w$ is the Schur polynomial $s_\lambda(x_1, \dots, x_r)$, where $\lambda = (w_r - r, \dots, w_2 - 2, w_1 - 1)$.}
\end{quote}

The above rule is extremely useful for identifying Schuberts as Schurs. For example, consider the Schubert polynomial corresponding to the permutation $[2\,4 \mid 1\,3]$. There is a descent at index $2$, and so $\lambda = (w_2 - 2, w_1 - 1) = (4 - 2, 2 - 1) = (2, 1)$ and
\[
    \mathfrak{S}_{[2\,4 \mid 1\,3]} = s_{(2, 1)}(x_1, x_2).
\]

The same rule applies for other permutations with one descent. For example, 
\[
    \mathfrak{S}_{[2 \mid 1\,3\,4]} = s_{(1)}(x_1).
\]
In particular, the rule also helps us realize that all Schur polynomials (of a finite number of variables) are Schubert polynomials; any Schur can be reconciled with a given permutation with one descent, which corresponds to a Schubert polynomial. \\

\textit{Additional Info: Schubert polynomials can be realized through pipe dreams or rc-graphs, perhaps a topic for further reading.}

\subsubsection{Bases for $H^* Fl(n)$ and Intro to Quantum Cohomology}

One basis for the cohomology ring $H^* Fl(n)$ is the set of Schubert polynomials corresponding to each permutation in $S_n$, or
\[
    \{ \mathfrak{S}_w(x_1, \dots, x_n) \mid w \in S_n \}.
\]

The set of Schubert polynomials is a positive basis; indeed, the Littlewood-Richardson Rule for Schur polynomials generalizes to Schuberts, and we have
\[
    \mathfrak{S}_u \mathfrak{S}_v = \sum_{w} c_{uv}^w \mathfrak{S}_w
\]
where each $c_{uv}^w$ is a unique (because the Schuberts are a basis) nonnegative coefficient. \\

Another well-known basis is known as the \textbf{standard elementary monomials}, of the form
\[
    \{e_{i_1}(1) \cdots e_{i_{n-1}}(n-1) \mid 0 \leq i_k \leq k \}
\]
where each $e_k(p)$ corresponds to the elementary symmetric polynomial of degree $k$ in $p$ variables $x_1, \dots, x_p$. Contrary to the previous Schubert basis, this is not a positive basis. For example, consider $Fl(3)$ and the product $e_1(1) \cdot e_1(1).$ We have that
\begin{align*}
    e_1(1) \cdot e_1(1) &= X_1^2 \\
    &= X_1(X_1 + X_2) - X_1X_2 \\
    &= e_1(1)e_1(2) - e_2(2).
\end{align*}

Note that each of these bases has precisely $n!$ entries, a defining characteristic for the bases of a complete flag manifold.

\textcolor{red}{[Discuss Quantum Schubert polynomials]}

\subsection{Quantum Grothendieck Polynomials}

\begin{definition}[Grothendieck Polynomials]
Let $\delta_i = \frac{f - s_i(f)}{x_i - x_{i+1}}$ be the difference operator, and $\pi_i = \delta_i(1-x_{i+1})$. 

Then, the Grothendieck polynomials are defined as:
\begin{align*}
    \mathfrak{G}_w(x_1, x_2, \ldots, x_n) &= \pi_{w^{-1}w_0}(x_1^{n-1} x_2^{n-2} \ldots x_{n-1}) 
\end{align*}
\end{definition}
I think some people have figured out \href{https://wiki.sagemath.org/combinat/MultivariatePolynomials}{how to compute these in sage}! Trying to figure out how the code works . . . 

Note that the lowest degree homogenous part of $\mathfrak{G}_w$ is given by $\mathfrak{S}_w$. 

\begin{theorem}[Grothendieck Pieri, from Lenart and Maeno]
    \begin{equation*}
        \mathfrak{G}_wg_p^k = \sum_\gamma m_p(\gamma) \mathfrak{G}_{\text{end}(\gamma)}
    \end{equation*} 
    where the sum is over all $k$-Pieri chain $\gamma$ on the infinite symmetric group that begin at $w$.

    We define $g_p^k = \mathfrak{G}_{c[k,p]} = \sum_{i=p}^k (-1)^{i-p} {{i-1}\choose{p-1}} e_i^k$. We let $\gamma$ denote a $k$-Pieri chain (see definition 2.14), $m_p(\gamma) = (-1)^{l(\gamma) -p}$ times the number of $p$-markings of $\gamma$.
\end{theorem}
The $k$-Pieri chain looks complicated — The monk formula covers the case when $p = 1$, and seems much nicer. I think the $k$-Pieri chain is the big (gross) thing that lets you generalize.

\begin{definition}[Grothendieck polynomial]
    The quantum Grothendieck polynomial $\mathfrak{G}_w^q$, for $w \in S_n$, is:
    \begin{equation*}
        \mathfrak{G}_w^q = \hat{Q}(\mathfrak{G}_w) \in \Z[q,x]
    \end{equation*}
\end{definition}
The quantum Grothendieck for $S_3$ are in example 3.19 of Lenart-Maeno. 

A combinatorial formula is given later in the paper.

\begin{theorem}[Quantum Monk for Grothendieck -- theorem 6.4 in LeM]

We have
\begin{equation*}
    \G_w^q \G_{s_k}^q = \sum_\pi (-1)^{l(\pi)- 1}q(\pi) \G_{end(\pi)} ; 
\end{equation*}
the summation is over all nonempty paths $\pi$ in the quantum $k$-Bruhat graph (of $S^\infty$) of the form 
\begin{equation*}
    w = w_0 \rightarrow w_1 \rightarrow \ldots \rightarrow w_s = \text{end}(\pi)
\end{equation*}
where $(a_1, b_1) \prec (a_2,b_2) \prec \ldots \prec (a_s,b_s)$.
\end{theorem}



