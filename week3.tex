\subsection{Weekly Summary}

Monday, June 3rd
\begin{itemize}
    \item Worked through Sheet 3 (Schubert polynomials introduction, Schubert polynomials in $Fl(3)$, relating Schubert polynomials $\mathfrak{S}$ in the basis of Schubert polynomials with partitions and Schur polynomials).
    \item Read potential projects background
\end{itemize}

Tuesday, June 4th
\begin{itemize}
    \item Met with Linda, talked about Sheet 3 (addressing sheet 3 problems and potential projects)
    \item Started working through potential projects warmup problems, including quantum Schubert polynomials and reading through Buch (Quantum Homology)
\end{itemize}

Wednesday, June 5th
\begin{itemize}
    \item Working on warmup problem exercise. Tried Quantum schubert multiplication with Quantum Pieri (not much success), found answer using Quantum Monk (need to check by hand)
    \item Reading Buch and Lenart-Maeno articles
\end{itemize}

Thursday, June 6th
\begin{itemize}
    \item Continue working on warmup problem exercise. Finalized Quantum Monk calculations for Sheet 3 Quantum Schubert cases
    \item Learned more about Grothendieck and quantum Grothendieck polynomials
    \item Begin working on project 2 problems (Chen, Tymoczko and LLMS as resources)
    \item Zoe and David met in the afternoon to discuss progress
\end{itemize}

Friday, June 7th
\begin{itemize}
    \item Learned more about Grothendieck (revisit/understand recursive definition and relation to Schuberts)
    \item Met with Linda
    \begin{itemize}
        \item Talked about quantum Grothendiecks -- everything is still given by quantizing the basis representation (given by definition 3.14 and 3.11 of Lenart Maeno). Way to get the quantization is by unwinding and doing 2-ish recursions, or using the unfriendly f map
        \item We should work through grothendiecks for $S^3$
        \item Also this probably isn't that bad to code -- can do this with the F's (Zoe will work on this and Linda will ask around)
        \item Bruhat graph -- tells you when one permutation is bigger than another. Quantum k-Bruhat order takes the quantum graph as defined and then adds one more condition
        \item History of project 1: Lenart Maeno define polynomials. Conjectured to be k-theory classes for flag varieties -- for a LONG time, no way to prove if this was or wasn't the case. Recently discovered this was the case, bc no one was sure if these were the right polynomials
        \item To do: Look at Grothendiecks for grassmannian case (nice combinatorial way, tableaux formula for grassmannian case). Look at quantum grothendiecks for $S^3$. Try to code quantum grothendiecks.
    \end{itemize}
\end{itemize}

\subsection{Points of Discussion for June 4th, Tuesday Meeting}

\begin{itemize}
    \item How can we determine a basis of Schubert polynomials for a general flag manifold?
    \item How can we reconcile the Schubert polynomials with the Schurs (which permutations correspond to which partitions)?
    \item For project 2: Which properties are we looking to preserve under the essential set analog?
\end{itemize}

\subsection{Points of Discussion for June 7th, Friday Meeting}

\begin{itemize}
    \item Regarding quantum Grothendieck polynomials, what is a $k$-Bruhat graph? Is this a nicer way to understand condition 3 from Lenart and Maeno? (which conditions correspond to partial flag, which ones correspond to quantum, which ones correspond to Monk vs. Pieri)
    \item How can we understand Grothendieck and (especially) quantum Grothendieck polynomials? 
    \item Discussion of Quantum Monk vs. Quantum Pieri and associated conditions for full flag (Buch page 6)

\end{itemize}






