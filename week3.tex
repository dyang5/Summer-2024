\subsection{Weekly Summary}

Monday, June 3rd
\begin{itemize}
    \item Worked through Sheet 3 (Schubert polynomials introduction, Schubert polynomials in $Fl(3)$, relating Schubert polynomials $\mathfrak{S}$ in the basis of Schubert polynomials with partitions and Schur polynomials).
    \item Read potential projects background
\end{itemize}

Tuesday, June 4th
\begin{itemize}
    \item Met with Linda, talked about Sheet 3 (addressing sheet 3 problems and potential projects)
    \item Started working through potential projects warmup problems, including quantum Schubert polynomials and reading through Buch (Quantum Homology)
\end{itemize}

\subsection{Points of Dicussion for June 4th, Tuesday Meeting}

\begin{itemize}
    \item How can we determine a basis of Schubert polynomials for a general flag manifold?
    \item How can we reconcile the Schubert polynomials with the Schurs (which permutations correspond to which partitions)?
    \item For project 2: Which properties are we looking to preserve under the essential set analog?
\end{itemize}

When we refer to the complete flag manifold $Fl(1, 2, 3, 4)$, observe that the Schubert polynomials correspond to permutations of $S_4$, and so $S_4$ is in some sense the basis for the flag manifold:
\[
    Fl(1, 2, 3, 4) \leftrightarrow S_4
\]
In general, for a flag manifold $Fl(n_1, n_2, \dots, n)$, we find that the ``basis'' corresponds to a subset of permutations of $S_n$:
\[
    Fl(n_1, n_2, \dots, n) \leftrightarrow \{ w \in S_n \mid w \text{ has descents } \subseteq \{ n_1, n_2, \dots, n\} \}
\]

Take the specific case of $Gr(2, 4) = Fl(2, 4)$. The basis corresponds to permutations 
$w \in S_4 \mid w \text{ has descent possibly at $2$}$; the permutations in said basis are simply $[1\, 2 \mid 3 \,4]$, $[1 3 \mid 2 4]$, $[1\, 3 \mid 2\,4]$, $[1\,4 \mid 2\,3]$, $[2\,3 \mid 1\,4]$, $[2\,4 \mid 1\,3]$, and $[3\,4 \mid 1\,2]$. \\

There is an association between the Schubert polynomials corresponding to each of these permutations and an associated Schur polynomial, characterized by the following rule: let $w = w_1 w_2 \dots w_n$. 
\textit{if $w_i < w_{i+1}$ for all $i \neq r$ (i.e. there is just one descent at index $r$), $\mathfrak{S}_w$ is the Schur polynomial $s_\lambda(x_1, \dots, x_r)$, where $\lambda = (w_r - r, \dots, w_2 - 2, w_1 - 1)$.}

