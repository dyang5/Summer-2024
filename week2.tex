
\subsection{Weekly Summary}

Tuesday, May 28th
\begin{itemize}
    \item Discussed Littlewood Richardson coefficients and practiced products of tableaux (Sheet 2 Problems 1-4)
    \item Schur polynomials are symmetric (see Jacobi-Trudi rule or Vandermonde determinant form and recall that transpositions generate $S_n$), form a basis for homogeneous symmetric polynomials
    \item Introduction to Schubert calculus and basics to Grassmanians
\end{itemize}

Wednesday, May 29th
\begin{itemize}
    \item Follow Dropbox resource and work through multiplication table of cohomology of Gr(2, 4) using dropbox resource
    \item Goal: reconcile algebraic/combinatorial methods (Pieri rules, Schur polynomials) with geometric Schubert conditions

\end{itemize}

Thursday, May 30th
\begin{itemize}
    \item Set up sage code to calculate Schur polynomials and Littlewood-Richardson coefficients
    \item Continue reading on Schubert varieties in flag manifolds (9.4), Relations between Schubert varieties, and Schubert polynomials (10.2 - 10.4, Fulton)
\end{itemize}

Friday, May 31st
\begin{itemize}
    \item Meeting to discuss Sheet 2 details (reconciling Schubert classes with geometric interpretations, multiplication tables for cohomologies, solving enumerative geometry problems)
    \item Work through remaining Sheet 2 Problems
\end{itemize}

\subsection{Points of Dicussion for May 31st, Friday Meeting}
\begin{itemize}
    \item Discussion of the geometric Schubert conditions corresponding to each Schubert cell
    
    \item We have an algebraic/combinatorial way to compute products in the multiplication table. How can we reconcile the algebraic methods with the geometric Schubert conditions?

    \item How can we use multiplication tables and Schubert classes to tackle Hilbert's problems (problem 6(b)-(d))? (revisit classic example of 4 lines)
\end{itemize}