\subsection{Weekly Summary}

Monday, June 17th
\begin{itemize}
    \item Attempt to work through products of Quantum Schuberts and Grothendiecks in Full Flags using George's code (not quite working)
    \item Start to look through Universal Schubert code and LaTeX
\end{itemize}

Tuesday, June 18th
\begin{itemize}
    \item Met with Linda to discuss progress on George's code and next steps for project
    \item Work with QK Maple calculator for cohomology, quantum cohomology, and quantum k-theory in full flags; our goal is to extend to partial flags
\end{itemize}

Meeting Notes
\begin{itemize}
    \item To do: use QK maple code, try to code and compute quantum Schuberts for partial flags, quantum grothendiecks for full flags. $Fl(3)$ is a good test case (for Weihong Xu's conjectures and for ones we come up with), as it is both a complete flag and a two-step flag. 
    \item Maple code has quantum Schubert. To get from equivariant to quantum, set $y_i = 0$ for each $y_i$.
    \item Goal: output everything for $Fl(3)$; $Fl(4)$ is also helpful. Then try $Fl(1,2,4)$ or $Fl((1,3,4)$.
    \item Tasks: code partial flag cohomology. Do universal grothendiecks and quantum grothendiecks for $Fl(3)$ and $Fl(4)$
\end{itemize}

Wednesday, June 19th
\begin{itemize}
    \item Finalize/contribute to George's code (adding quantum Schubert basis function)
    \item Begin to adapt QKcalc code to quantum Schuberts in partial flags
\end{itemize}

Thursday, June 19th
\begin{itemize}
    \item Typed up meeting notes, reviewing $\sigma$ matrix approach for partial flags 
    \item Started setting up $A_n^q$ matrix, per Ciochan-Fontanine, which will be used to calculate quantized standard elementary monomials in partial flags
\end{itemize}