\subsection{Weekly Summary}

Monday, June 10th
\begin{itemize}
    \item Coded our own implementation of Grothendieck polynomials, with isobaric divided difference operator, in Sage
    \item Begin looking into connections between Schurs, Schuberts, and Grothendiecks
\end{itemize}

Tuesday, June 11th
\begin{itemize}
    \item Met with Linda, discussed Quantum Schuberts in full flags (matrix determinant approach and with Dynkin diagrams)
    \item Formalized project direction (work on understanding Quantum Schuberts in partial flags, extend to defining Quantum Grothendiecks in partial flags, Monk formula)
\end{itemize}

Wednesday, June 12th
\begin{itemize}
    \item Typed notes up from yesterday (standard elementary monomials using matrix and Dynkin diagram approaches)
    \item Continue reading through FGP and Ciochan-Fontanine papers for Quantum Schuberts in partial flags
\end{itemize}

\subsection{Points of Discussion for June 14th, Friday Meeting}

\begin{itemize}
    \item For partial flags do we always need the largest space i.e. if we are working in $\C^4$, does a partial flag always include a dimension of size $4$: $Fl(1, 2, \C^4)$?
    \item Work through Schubert polynomial difference rule (check order of divided difference, Schubert representations)
    \item Giambelli polynomials/Giambelli Formula (expressing one Schubert class in terms of others?) and its relation to Grothendiecks and Pieri Formula (mentioned in Buch QH)
    \item Products of Quantum Schuberts in Partial Flags (discuss George's code)

\end{itemize}

\subsection{Reconciling Schurs, Schuberts, and Grothendiecks}

Schur polynomials for $Gr(k,n)$ can be written as a sum of monomials corresponding to skew-symmetric Young Tableau on $\lambda$ filled by $1, 2, \dots, k$. \\

Schubert polynomials for $Fl(n)$ can be obtained via divided difference operator, or can be written as a sum of monomials corresponding to reduced pipe dreams/RC-graphs, see \href{https://sites.math.washington.edu/~billey/papers/bjs.pdf}{Billey-Jockusch-Stanley}. \\

The stable Grothendieck polynomials, a special class of Grothendieck polynomials, for $Gr(k,n)$ can be written as a sum of monomials corresponding to skew-symmetric set-valued tableau on $\lambda$ filled by $1, 2, \dots, k$, see 
\href{https://sites.math.rutgers.edu/~asbuch/papers/combkth.pdf}{Buch}. The stable Grothendecieks are defined as a limit
\[ 
    G_w(x) = \lim\limits_{k \rightarrow \infty}\mathfrak{G}_{1^k \times w}(x) = \lim\limits_{k \rightarrow \infty}\mathfrak{G}_{[1, \, \dots \,, \, k, \, w(1) + k \,, \,w(2) + k \,, \, \dots \,, \,w(n)+ k]}(x).
\]

Grothendieck polynomials for $Fl(n)$ can be obtained via divided difference operator, and are related to bumpless pipe dreams, see \href{https://www.symmetricfunctions.com/grothendieck.htm}{Symmetric Functions website}.
