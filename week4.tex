\subsection{Weekly Summary}

Monday, June 10th
\begin{itemize}
    \item Coded our own implementation of Grothendieck polynomials, with isobaric divided difference operator, in Sage
    \item Begin looking into connections between Schurs, Schuberts, and Grothendiecks
\end{itemize}

Tuesday, June 11th
\begin{itemize}
    \item Met with Linda, discussed Quantum Schuberts in full flags (matrix determinant approach and with Dynkin diagrams)
    \item Formalized project direction (work on understanding Quantum Schuberts in partial flags, extend to defining Quantum Grothendiecks in partial flags, Monk formula)
\end{itemize}

Wednesday, June 12th
\begin{itemize}
    \item Typed notes up from yesterday (standard elementary monomials using matrix and Dynkin diagram approaches)
    \item Continue reading through FGP and Ciochan-Fontanine papers for Quantum Schuberts in partial flags
\end{itemize}

Thursday, June 13th
\begin{itemize}
    \item Started experimenting with George's code for Quantum Schuberts
    \item Finalize questions for meetings (see below)
\end{itemize}

Friday, June 14th
\begin{itemize}
    \item Met with Linda, discussed quantization and Kato maps, Schubert polynomial difference rules, partial flag cases
    \item Debugged George's code for Quantum Schuberts, now experimenting more
    \item Goal: understand partial flag Quantum Schubert cases
\end{itemize}

\subsection{Points of Discussion for June 14th, Friday Meeting}

\begin{itemize}
    \item Work through Schubert polynomial difference rule (check order of divided difference, Schubert representations)
    \item Giambelli polynomials/Giambelli Formula (expressing one Schubert class in terms of others?) and its relation to Grothendiecks and Pieri Formula (mentioned in Buch QH)
    \item Products of Quantum Schuberts in Partial Flags (discuss George's code). in partial flag case is it just that we have different bases?
    \item Can we go through the arrow diagram and what parts do and don't carry over to the partial flag case?
    \item Lenart Maeno says "We use
a new quantization map (Definition 3.14, Corollary 5.7), which is based on the presentation of QK(Fln)
in [22], and which has a factorization property similar to that of the cohomology quantization map. Thus, our quantum Grothendieck polynomials are different from those in [19], which were defined by
applying the cohomology quantization map to Grothendieck polynomials." What does this mean?? What is the benefit of using one construction over the other?
\item Messing around with quantum Grothendieck data - what should I be looking for?
\item Lenart Maeno say 4.1 is related to the Grassmannian . . . how? presumably that's something we want to look at . . . 
\end{itemize}

Meeting Notes
\begin{itemize}
    \item Buch QH partial flag has same results as CF
    \item Stick to Monk/Chevalley
    \item Sage Quantum Bruhat graph
    \item Schubert calculus: 
    \begin{enumerate}
        \item How do you write it as a polynomial ring?
        \item How do you write the basis as polynomials? -- Giambelli
        \item Multiplication rule (Monk, Pieri)
    \end{enumerate}
    \item Flag 1,3,4 table in CF is good. Get more familiar w sigmas
    \item Look at some products (specifically multiplying by a simple transposition). Write it in terms of a basis 
    \item Get more comfortable with Giameblli for partial flags, see if you can understand some of the code (for the Monk) -- do small cases for the quantum schubert partial flags, and quantum grothendieck for full flags
    \item Try to modify write polynomial in terms of basis for different partial flags
\end{itemize}

\subsection{Reconciling Schurs, Schuberts, and Grothendiecks}

Schur polynomials for $Gr(k,n)$ can be written as a sum of monomials corresponding to skew-symmetric Young Tableau on $\lambda$ filled by $1, 2, \dots, k$. \\

Schubert polynomials for $Fl(n)$ can be obtained via divided difference operator, or can be written as a sum of monomials corresponding to reduced pipe dreams/RC-graphs, see \href{https://sites.math.washington.edu/~billey/papers/bjs.pdf}{Billey-Jockusch-Stanley}. \\

The stable Grothendieck polynomials, a special class of Grothendieck polynomials, for $Gr(k,n)$ can be written as a sum of monomials corresponding to skew-symmetric set-valued tableau on $\lambda$ filled by $1, 2, \dots, k$, see 
\href{https://sites.math.rutgers.edu/~asbuch/papers/combkth.pdf}{Buch}. The stable Grothendecieks are defined as a limit
\[ 
    G_w(x) = \lim\limits_{k \rightarrow \infty}\mathfrak{G}_{1^k \times w}(x) = \lim\limits_{k \rightarrow \infty}\mathfrak{G}_{[1, \, \dots \,, \, k, \, w(1) + k \,, \,w(2) + k \,, \, \dots \,, \,w(n)+ k]}(x).
\]

Grothendieck polynomials for $Fl(n)$ can be obtained via divided difference operator, and are related to bumpless pipe dreams, see \href{https://www.symmetricfunctions.com/grothendieck.htm}{Symmetric Functions website}.
