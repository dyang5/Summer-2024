\documentclass[11pt]{article}
\usepackage{styletemplate}

\setlength{\parindent}{0pt}
\begin{document}

\title{\textbf{Maple Code Documentation}}
\author{David Yang \and Zoe Markman}
\date{Summer 2024}

\maketitle

For all code testing, be sure to import the proper files (\texttt{modified\_qkcalc.txt}, \texttt{all\_working.txt}, and \texttt{QKChevTypeA.txt}, \texttt{grothendieck\_testing.txt}). \\

To define a particular flag, define and specify an array, as follows
\begin{lstlisting}[language = Maple]
# define A as Fl(1, 3, 4)
local A := Array(1..3, [1, 3, 4]):
\end{lstlisting}

You will need to pass your defined flag into each function that requires an Array as a parameter to specify the partial flag you are working under.

\section{Testing Quantum Grothendieck Conjectures}

\section{Quantum Grothendiecks}
    \subsection{General Commands}
    \begin{itemize}
        \item \texttt{qkmonk\_GP\_A\_table($F^*$)} prints the Monk/Chevalley relations for flag $F$. \textit{note: $F^*$ refers to the fact that the flag $F$ must be written with a leading zero, so $Gr(2, 4) = [0, 2, 4]$}
        \begin{itemize}
            \item \texttt{qkmonk\_GP\_A\_table($[0, 2, 4]$)} gives the Chevalley relations for $Gr(2, 4)$. 
            \item imported from \texttt{QKChevTypeA.txt} code
        \end{itemize}
        \item \texttt{check\_relation(expr, ideal)} checks that an expression is $0$ in the given ideal
        \item \texttt{ggp(A, perm)} returns the quantum grothendieck polynomial corresponding to perm in flag $A$
    \end{itemize}
    
    \subsection{Conjecture Testing}
    
    \subsection{Manual Guessing}

    \begin{itemize}
        \item f\_manual, qf\_manual\_guess, qe\_bar\_guess
    \end{itemize}


\section{Quantum Schuberts}
\begin{itemize}
    \item \texttt{usp(perm)} returns the universal schubert polynomial for perm in the full flag
    \item \texttt{sp(perm)} returns the quantum schubert polynomial of perm in the full flag
    \item \texttt{\_sp(A, perm)} returns the quantum schubert polynomial corresponding to perm in flag $A$
    \item \texttt{print\_all\_qs(A)} prints all quantum schuberts in the basis of the quantum cohomology of flag $A$
    \item \texttt{qs\_pf\_basis(A)} returns the basis of quantum schuberts in the quantum cohomology of flag $A$
    \item \texttt{check\_cioFonFlag\_relations($A$)} confirms that in $A = Fl(1, 3, 4)$, quantum Schuberts match quantum Giambelli polynomials from Table 1 in Ciocan-Fontanine
\end{itemize}

\end{document}