\subsection{Weekly Summary}

Monday, June 17th
\begin{itemize}
    \item Attempt to work through products of Quantum Schuberts and Grothendiecks in Full Flags using George's code (not quite working)
    \item Start to look through Universal Schubert code and LaTeX
\end{itemize}

Tuesday, June 18th
\begin{itemize}
    \item Met with Linda to discuss progress on George's code and next steps for project
    \item Work with QK Maple calculator for cohomology, quantum cohomology, and quantum k-theory in full flags; our goal is to extend to partial flags
\end{itemize}

Meeting Notes
\begin{itemize}
    \item To do: use QK maple code, try to code and compute quantum Schuberts for partial flags, quantum grothendiecks for full flags. $Fl(3)$ is a good test case (for Weihong Xu's conjectures and for ones we come up with), as it is both a complete flag and a two-step flag. 
    \item Maple code has quantum Schubert. To get from equivariant to quantum, set $y_i = 0$ for each $y_i$.
    \item Goal: output everything for $Fl(3)$; $Fl(4)$ is also helpful. Then try $Fl(1,2,4)$ or $Fl((1,3,4)$.
    \item Tasks: code partial flag cohomology. Do universal grothendiecks and quantum grothendiecks for $Fl(3)$ and $Fl(4)$
\end{itemize}

Wednesday, June 19th
\begin{itemize}
    \item Finalize/contribute to George's code (adding quantum Schubert basis function)
    \item Begin to adapt QKcalc code to quantum Schuberts in partial flags
\end{itemize}

Thursday, June 19th
\begin{itemize}
    \item Typed up meeting notes, reviewing $\sigma$ matrix approach for partial flags 
    \item Started setting up $A_n^q$ matrix, per Ciochan-Fontanine, which will be used to calculate quantized standard elementary monomials in partial flags
\end{itemize}

Friday, June 20th
\begin{itemize}
    \item Met with Linda, discussed next steps for coding quantum schuberts in partial flags (use universal schubert polynomials/double schubert polynomials or combinatorial methods)
    \item Completed quantum elementary standard monomial code and fixed some errors in Dynkin matrix examples
\end{itemize}


Point of Discussion for Friday, June 20th Meeting
\begin{itemize}
    \item Revisit quantum Giambelli classes and how to use the matrix to determine Quantum Schuberts in partial flags
    \item Review Ciocan-Fontanine formulas for $a$ and $b$ matrices (485, 487) in partial flag cases
\end{itemize}


Meeting Notes
\begin{itemize}
    \item To do: use QK maple code, try to code and compute quantum Schuberts for partial flags, quantum grothendiecks for full flags. $Fl(3)$ is a good test case (for Weihong Xu's conjectures and for ones we come up with), as it is both a complete flag and a two-step flag. 
    \item Maple code has quantum Schubert. To get from equivariant to quantum, set $y_i = 0$ for each $y_i$.
    \item Goal: output everything for $Fl(3)$; $Fl(4)$ is also helpful. Then try $Fl(1,2,4)$ or $Fl((1,3,4)$.
    \item Tasks: code partial flag cohomology. Do universal grothendiecks and quantum grothendiecks for $Fl(3)$ and $Fl(4)$
\end{itemize}

6/21 Notes:
\begin{itemize}
    \item typo in CF
    \item Create elementary symmetric polynomials (which we can now do). Then create quantum schuberts
    \item elementary symmetrics go by step. (e.g. for 1,2,4 $e_1^q(3)$ refers to the step 4)
    \item Python code has some divided difference code
    \item Tasks:
    \begin{enumerate}
        \item Create quantum elementary symmetrics
        \item Goal: find quantum schubert polynomials. Take word of longest length, double, divided difference (the right side does $w_0$ to $w$. The left hand side does $w$ to $w_0$) 
        \item Play around with Weihong's code, and familiarize ourselves with $Fl(1,2,4)$ and $Fl(1,3,4)$
    \end{enumerate}
\end{itemize}

6/24 Notes:
\begin{itemize}
    \item Worked on coding. A lot.
    \item Zoe to do: Figured out how to find the $u's$ and $v's$. Now wants to write schubert polynomials in terms of x and y using qk calc code.
    \begin{itemize}
        \item Game plan: create new function adjusted from sp function that takes in the variable letter as an input. Then spits out the normal schubert expression that we want but in $x$ or $y$ depending
    \end{itemize}
    \item Or can do python code, setting almost everything to 0
\end{itemize}